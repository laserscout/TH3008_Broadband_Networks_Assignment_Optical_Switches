\section{Δίκτυα εντός ολοκληρωμένων και πλακετών}

Μία απρόσμενη χρήση των οπτικών δικτύων και κατ' επέκταση ανάγκη για
οπτική μεταγωγή είναι στις εσωτερικές διασυνδέσεις των ολοκληρωμένων
κυκλωμάτων. Η απόδοση πολυπύρηνων επεξεργαστών εξαρτάται πέρα από την
απόδοση του κάθε πυρήνα ξεχωριστά, και από την απόδοση τις
επικοινωνίας των πυρήνων αυτών. Με την συμπύκνωση πολλών πυρήνων πάνω
σε ένα ολοκληρωμένο, ο τόπος διασύνδεσης έχει αλλάξει από μέσω
δίαυλου, σε δίκτυα \cite{cpu-nano-interconnect},
\cite{cpu-hp-switches}. Γι' αυτόν τον λόγο ο καθοριστικός παράγοντας
ποια για την ολική απόδοση ενός ολοκληρωμένου είναι τα χαρακτηριστικά
της διασύνδεση του, η οποία όταν ακολουθεί μία συμβατική, ηλεκτρική
σχεδίαση εισάγει μεγάλη κατανάλωση, άρα και θερμική έξοδο,
περιορισμένη ταχύτητα και υψηλή καθυστέρησή. Οπότε η χρήση οπτικών
δικτύων για την διασύνδεση πάνω στα ολοκληρωμένα, θεωρείτε μια πιθανή
λύση \\cite{cpu-optical-requirements}.

Πολλαπλές τοπολογίες δικτύου και μεθόδους μεταγωγής δοκιμάζονται στην
βιβλιογραφία αναλόγως με τις ανάγκες τις κάθε υλοποίησης. Η ερευνητική
κοινότητα φαίνεται να επικεντρώνεται στην τοπολογία πλέγματος λόγο της
συμμετρίας του και του απλού αλγορίθμου δρομολόγησης. Στοιχεία τέτοιων
δικτύων μπορούν να είναι 5-θυρών οπτικοί δρομολογητές χτισμένοι πάνω
στο πυρίτιο οι οποίοι χρησιμοποιούν συμβατικές τεχνολογίες μεταγωγών
κυκλωμάτων \cite{cpu-microring}

Πολυπλοκότερες διατάξεις είναι απαραίτητες όταν η διασύνδεση πρέπει να
ακολουθεί κάποιο ποιο πολύπλοκο πρωτόκολλο όπως το Ethernet, άρα και
υπάρχει η ανάγκη για μεταγωγή πακέτων. Στην περίπτωση των διασυνδέσεων
εντός ενός ολοκληρωμένου, ή εντός μιας πλακέτας, η ανάγκη ενταμιευτών
μπορεί να αποφευχθεί και η χρήση ενός αποκλειστικά οπτικού μεταγωγέα
πολύ υψηλής ρυθμαπόδοσης έχει πολύ καλή συμπεριφορά.\cite{6359293}

%%% Local Variables:
%%% mode: latex
%%% TeX-master: "main"
%%% End:
