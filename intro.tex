\section{Εισαγωγή}

Σε ένα οπτικό δίκτυο για να είναι εφικτή η δυναμική δρομολόγηση
υπάρχει η ανάγκη ενός μεταγωγέα (switch) ο οποίος θα ανακατευθύνει την
δέσμη φωτός από μία θύρα εισόδου, στην επιθυμητή θύρα εξόδου. Όπως και
στα ηλεκτρικά σήματα, και στα οπτικά υπάρχουν διάφορα επίπεδα
ιεραρχίας στην μεταγωγής του σήματος, από τη κυρίως στατική μεταγωγή
κυκλώματος όπου προσφέρει μια αργά μεταβαλλόμενη ένα προς ένα σύνδεση
δύο οπτικών τερματικών, στην μεταγωγή πακέτων όπου ο χρόνος μεταγωγής
πρέπει να είναι τάξης μεγέθους μικρότερος από τον χρόνο μετάδοσης του
κάθε πακέτου. Υπάρχει και μία ενδιάμεση μεταγωγή ριπών όπου εκ των
προτέρων και σε ξεχωριστό κανάλι στέλνεται η πληροφορία δρομολόγησης
και τα πακέτα δεδομένων μεταδίδονται σε ριπές. Οι διαφορετικές αυτές
μεθόδους μεταγωγής είναι πολύ διαφορετικές μεταξύ τους και θα
μελετηθούν ξεχωριστά.


Στην εργασία του El-Bawab και Shin \cite{} δόθηκε μία επισκόπηση των
τεχνολογιών της εποχής όπου επιτρέπουν ή και μελετάνε την μεταγωγή
πακέτων στο οπτικό πεδίο. Τέτοιες τεχνολογίες υπάρχουν πολλές αλλά
είναι σε ερευνητικό στάδιο και υπάρχουν κάποια εμπόδια τα οποία πρέπει
να ξεπεραστούν ώστε να γίνει εφικτή η εμπορική χρήση των οπτικών
μεταγωγών. Τα σημαντικότερα εμπόδια είναι η εκταμίευση και η
επεξεργασία οπτικών σημάτων στα μικρά χρονικά διαστήματα των
πακέτων. Με την χρήση καθρεπτών, κρυστάλλων διάθλασης και ολικής
ανάκλησης, είναι εφικτή με τις σημερινές τεχνολογίες η ανακατεύθυνση
της δέσμης του φωτεινού σήματος, όπως και η πολλυπλεξία δεσμών
διαφόρων δεσμών διαφορετικών μηκών κύματος σε μία. Όμως οι τεχνολογίες
αποθήκευσης της οπτικής πληροφορίας δεν έχουν την ίδια
αποτελεσματικότητα της ηλεκτρικής σε DRAM και είναι αρκετά
πειραματικές και πολύπλοκες. Επίσης η εξασθένιση της αποθηκευμένης
πληροφορίας είναι πολύ γρήγορη και περιορίζει περαιτέρω την ποσότητα
και την διάρκεια για την οποία η πληροφορία μπορεί να
εκταμιευθεί. Τέλος ενώ έχει υπάρξει πρόοδος τους τομείς της
αναγέννησης του σήματος, του συγχρονισμού, μετατροπής μήκους κύματος
κ.α. η περιορισμένη επεκτασιμότητα και το αυξημένο κόστος αυτών των
τεχνολογιών δεν επιτρέπουν την ευρεία χρήση τους σε εμπορικές
εφαρμογές.



Στην μεταγωγή κυκλώματος, η οποία υπάρχει και εμπορικά,
χρησιμοποιώντας μεθόδους μεταγωγής όπως MEMS etc\ldots
% ! Θέλει συμπλήρωμα...


Η οπτική μεταγωγή όπως προαναφέρθηκε μπορεί να χωριστεί στους δύο
παρόμοιους τύπους, την οπτική μεταγωγή πακέτων (OPS) όπου η
δρομολόγησή γίνεται σε επίπεδο πακέτων, και την οπτική μεταγωγή ριπών
(OBS) όπου μεγαλύτερες οντότητες οι οποίες περιέχουν πολλαπλά πακέτα
και ονομάζονται "ριπές" δρομολογούνται. Η "ριπή" θυμίζει το frame του
ATM πρωτοκόλλου.

Το επίπεδο των πακέτων για την οπτική μεταγωγή φαίνεται να είναι η
ποιο απλή και φυσική επιλογή για δρομολόγηση. Όμως ο είναι δύσκολο να
εφαρμοστούν οι στατιστικοί κανόνες μεγεθών ενταμιευτών για την ομαλή
λειτουργία ελέγχου κίνησης (TCP congestion control). Η οπτική μεταγωγή
πακέτων, προσπαθώντας να ξεπεράσει αυτόν τον περιορισμό, μπορεί να
χρησιμοποιεί επιπλέον τεχνικές για την διαχείριση πακέτων σε συνθήκες
υψηλού φορτίου. Μία μπορεί να είναι η μετατροπή μήκους κύματος, όπου
αν εμφανιστεί ένα πακέτο σε έναν μεταγωγέα όπου οι ενταμιευτές του
είναι γεμάτοι και η θύρα εξόδου κατειλημμένη, να μετατρέψει το μήκος
κύματος αυτού του πακέτου και να το αποστείλει στην ίδια θύρα
εξόδου. Μία άλλη είναι αποστολή του πακέτου μέσω μιας δευτερεύουσας
διαδρομής από μια θύρα εξόδου όπου δεν έχει συνθήκες υψηλού
τηλεπικοινωνιακού φορτίου. Τέτοιες μέθοδοι κατά πρώτων δεν επαρκούν
για να καλύψουν ικανοποιητικά την έλλειψή αρκετά μεγάλων ενταμιευτών
και κατά δεύτερο μπορεί να δημιουργούν επιπλέον προβλήματα από αυτά
όπου μπορούν να λύσουν.

Η οπτική μεταγωγή ριπών είχε προταθεί \cite{} και είναι ένα υβριδικό
της οπτικής μεταγωγής κυκλώματος και της οπτικής μεταγωγής πακέτων. Με
την χρήση ηλεκτρικών ενταμιευτών στις άκρες του οπτικού δικτύου,
δίνεται η δυνατότητα της ένα προς ένα σύνδεση για την αποστολή ενός
συνόλου πακέτων. Σε γενικές γραμμές πριν την αποστολή της ριπής
ανταλλάσσονται μηνύματα δρομολόγησης, μέσω άλλου σήματος ή και άλλου
μέσου, τα οποία δημιουργούν για ένα κλειστό ή ανοιχτό χρονικό διάστημα
την επιθυμητή σύνδεση. Αν η δημιουργία της σύνδεσης είναι επιτυχής,
λαμβάνει χώρα η αποστολή της ριπής και έπειτα καταστρέφεται η σύνδεση,
ενώ αν είναι ανεπιτυχής, τα δεδομένα παραμένουν στον ενταμιευτή και
επιχειρείται εκ νέου μετά από κάποιο χρονικό διάστημα η αποστολή
τους. Αυτή η μέθοδος μπορεί να δημιουργεί ανεπιθύμητες επιπτώσεις σε
ένα TCP δίκτυο, ειδικά όταν το μέγεθος της ριπής είναι πολύ μεγάλο και
υπάρχουν πολλαπλά διαδοχικά TCP πακέτα εντός μίας ριπής. Αυτό
συμβαίνει γιατί για το TCP πρωτοκόλλο, τρεις διαδοχικές αποτυχίες
αποστολής είναι ένα σημάδι υψηλής τηλεπικοινωνιακής κίνησης και ο
ρυθμός αποστολής αυτόματα μειώνεται.

Αν επιλεχθεί η χρήση ηλεκτρικών ενταμιευτών και την μεταγωγή των
πακέτων μετά από μετατροπή του οπτικού σήματος σε ηλεκτρικό λύνεται το
πρόβλημα την αποθήκευσης των πακέτων ενώ η ρυθμαπόδοση παραμένει υψηλά
λόγου της χρήσης των οπτικών ινών ως το μέσο μετάδοσης. Η μετατροπή
όμως του οπτικού σήματος σε ηλεκτρικό είναι ενεργοβόρα σε τέτοιο βαθμό
όπου λόγο των θερμικών απωλειών η πυκνότητά του εξοπλισμού μεταγωγής
δεν μπορεί να ξεπεράσει έναν συγκεκριμένο βαθμό. Η ολοένα αύξηση των
ταχυτήτων των δικτύων έχει δημιουργήσει την ανάγκη για συστήματα
μεταγωγής με πολλαπλά racks για την ικανοποιητική ψύξη τους. Πέρα
από τον εμφανή χωρικό περιορισμό που εισάγει η υλοποίηση αυτή, εισάγει
περιορισμούς ταχύτητας και καθυστέρησης λόγο της απαραίτητης
εξωτερικής διασύνδεσης των πολλαπλών ντουλαπών.

Η ανάγκη για ένα σύστημα μεταγωγής πακέτων οπτικών δικτύων η οποία θα
άρει αυτούς τους περιορισμούς είναι εμφανής και έχει αναγνωριστεί από
διάφορα κράτη τα οποία έχουν προσφέρει επιχορηγήσεις σε διάφορα
σχετικά ερευνητικά προγράμματα όπως: τα Ευρωπαϊκά KEOPS \cite{736580},
DAVID \cite{1230193}, BOOM \cite{5398841} και ECOFRAME \cite{5465485}
και τα Αμερικάνικα LASOR \cite{4804163} και IRIS \cite{IRIS}. Στην
παρούσα εργασία θα παρουσιαστούν κάποια από αυτά.

%%% Local Variables:
%%% mode: latex
%%% TeX-master: "main"
%%% End:
