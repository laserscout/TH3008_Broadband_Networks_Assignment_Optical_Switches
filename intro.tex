\section{Εισαγωγή}

Όπως και στα ηλεκτρικά σήματα, και στα οπτικά υπάρχουν διάφορα επίπεδα
ιεραρχίας στην μεταγωγής του σήματος, από τη κυρίως στατική μεταγωγή
κυκλώματος όπου προσφέρει μια αργά μεταβαλλόμενη ένα προς ένα σύνδεση
δύο οπτικών τερματικών, στην μεταγωγή πακέτων όπου ο χρόνος
μεταγωγής πρέπει να είναι τάξης μεγέθους μικρότερος από τον χρόνο
μετάδοσης του κάθε πακέτου και την ενδιάμεση μεταγωγή ριπών όπου εκ
των προτέρων και σε ξεχωριστό κανάλι στέλνεται η πληροφορία
δρομολόγησης και τα πακέτα δεδομένων μεταδίδονται σε ριπές. Οι
διαφορετικές αυτές μεθόδους μεταγωγής είναι πολύ διαφορετικές μεταξύ
τους και πρέπει να μελετηθούν ξεχωριστά.

Στην μεταγωγή κυκλώματος, η οποία υπάρχει και εμπορικά,
χρησιμοποιώντας μεθόδους μεταγωγής όπως MEMS etc\ldots
% ! Θέλει συμπλήρωμα...

Η μεταγωγή πακέτων οπτικών δικτύων κατά κύριο λόγο γίνεται με την
μετατροπή του οπτικού σήματος σε ηλεκτρικό και η μεταγωγή γίνεται στο
ηλεκτρικό σήμα. Η μετατροπή αυτή όμως είναι ενεργοβόρα σε τέτοιο βαθμό
όπου λόγο των θερμικών απωλειών η πυκνότητά του εξοπλισμού μεταγωγής
δεν μπορεί να ξεπεράσει έναν συγκεκριμένο βαθμό. Η ολοένα αύξηση των
ταχυτήτων των δικτύων έχει δημιουργήσει την ανάγκη για συστήματα
μεταγωγής με πολλαπλές ντουλάπες για την ικανοποιητική ψύξη τους. Πέρα
από τον εμφανή χωρικό περιορισμό που εισάγει η υλοποίηση αυτή, εισάγει
περιορισμούς ταχύτητας και καθυστέρησης λόγο της απαραίτητης
εξωτερικής διασύνδεσης των πολλαπλών ντουλαπών. Η ανάγκη για ένα
σύστημα μεταγωγής πακέτων οπτικών δικτύων η οποία θα άρει αυτούς τους
περιορισμούς είναι εμφανής και έχει αναγνωριστεί από διάφορα κράτη τα
οποία έχουν προσφέρει επιχορηγήσεις σε διάφορα σχετικά ερευνητικά
προγράμματα όπως: τα Ευρωπαϊκά KEOPS \cite{736580}, DAVID
\cite{1230193}, BOOM \cite{5398841} και ECOFRAME \cite{5465485} και τα
Αμερικάνικα LASOR \cite{4804163} και IRIS \cite{IRIS}.

Η ενδιάμεση λύση είναι η μεταγωγή ριπών κτλ\ldots
% ! Θέλει συμπλήρωμα...

%%% Local Variables:
%%% mode: latex
%%% TeX-master: "main"
%%% End:
